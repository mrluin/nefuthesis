% !Mode:: "TeX:UTF-8"
% !TEX builder = LATEXMK
% !TEX program = xelatex
% !Template main file/ Created By TTB finished in 2019.07.11


% 该模板使用的选项:
% 学硕(Academic Master)请选择 AcaMaster,专硕(Professional Master)请选择 ProMaster,
% 如果有合作导师请选择 cpsupervisor,如果没有合作导师请选择 nocpsupervisor
\documentclass[AcaMaster, cpsupervisor]{nefuthesis}

% 欢迎大家帮忙核对字体以及格式,如果疏漏欢迎提出issue,欢迎PR https://github.com/mrluin/nefuthesis

% 论文中文标题
\title{用\LaTeX 使论文变得优雅} 
% 论文英文标题
\englishtitle{Make Paper Elegant by \LaTeX}
% 你的名字
\author{狗蛋儿}
\authorenglishname{Guldan}
% 学校代码
\serialnumber{xxxxx} % 这个内容先不用填写
% 学号
\studentnumber{xxxxxx} % 这个内容也一样不用填写
% 指导教师的中文名字,职位,以及学校
% 指导教师的英文名字,职位  副教授 Associate Prof. 教授 Prof.
% 四号字体用14pt进行占格,词与词之间用5pt进行间隔
% 仅作为例子进行参考
\supervisor{李老板\hspace{5pt}副教授\hspace{5pt}东北林业大学}
\supervisorenglish{Associate Prof.Laoban Li}
% 合作导师,如果没有合作导师,就在\documentclass选项栏中加上"nocpsupervisor"。
\cpsupervisor{王老板\hspace{5pt}教\hspace{14pt}授\hspace{5pt}东北林业大学}
\cpsupervisorenglish{Associate Prof.Laoban Wang}
% 申请学位级别 此处 \hspace{14pt} 为了与前文中三个字的内容进行对齐
\degreelevel{硕\hspace{14pt}士}
\degreelevelenglish{Master}
% 学科专业
\major{计算机系统结构}
\majorenglish{Computer Architecture}
% 论文提交日期
\submitdate{2019.06}
% 论文答辩日期
\replydate{2019.06}
\replydateenglish{June, 2019}
% 授予学位单位
\awardingunit{东北林业大学}
\awardingunitenglish{Northeast Forestry University}
% 授予学位日期
\awardingdate{2019.06}

\begin{document}
% 封面页 中文封面 英文封面 
\maketitle  
\frontmatter
% 中文摘要 英文摘要
\begin{cabstract}
	这里是中文摘要部分,标题格式为加粗黑体小二,内容格式为宋体小四
	
	我是来测试格式的
	
\end{cabstract}


% 注意关键词与关键词之间用中文逗号进行间隔
\ckeywords{关键词1;关键词2;关键词3;关键词4}
\begin{eabstract}
	Here is English abstract part, the font style of title is blond Times New Roman xiaoer, the font style of content is Times New Roman xiaosi
	
	Testtest
\end{eabstract}


% 注意关键词与关键词之间用英文逗号进行间隔
\ekeywords{keyword1;keyword2;keyword3;keyword4}
% 目录
\tableofcontents
\mainmatter
% 正文
% !TEX root = ../thesis.tex

\chapter{对基础性内容进行介绍}
首先对基础性功能进行简单的说明,如有详细的内容需要了解,大家仍需多进行google,在社区上寻找答案,因为我也是本着学习的态度来整理这份模板哒。

对于脚注的使用{\footnote{我是正文中的脚注}}

\section{列表环境测试}
列表交叉引用的情况\ref{itm:11},\ref{itm:12},\ref{itm:13},\ref{itm:14},这里相信以大家聪明的大脑一看就能明白。

\begin{enumerate}
	\item 第一级列表\label{itm:11}
	\item 第一级列表
	\begin{enumerate}
		\item 第二级列表\label{itm:12}
		\item 第二级列表
		\begin{enumerate}
			\item 第三级列表\label{itm:13}
			\item 第三级列表
			\begin{enumerate}
				\item 第四级列表\label{itm:14}
				\item 第四级列表
				\item 第四级列表
				\item 第四级列表
			\end{enumerate}
			\item 第三级列表
			\item 第三级列表
			\item 第三级列表
		\end{enumerate}
		\item 第二级列表
		\item 第二级列表
	\end{enumerate}
	\item 第一级列表
	\item 第一级列表
	\item 第一级列表
\end{enumerate}
测试另外一种列表环境
\begin{itemize}
	\item 第一级列表
	\item 第一级列表
	\begin{itemize}
		\item 第二级列表
		\item 第二级列表
		\begin{itemize}
			\item 第三级列表
			\item 第三级列表
			\begin{itemize}
				\item 第四级列表
				\item 第四级列表
				\item 第四级列表
				\item 第四级列表
			\end{itemize}
			\item 第三级列表
			\item 第三级列表
			\item 第三级列表
		\end{itemize}
		\item 第二级列表
		\item 第二级列表
	\end{itemize}
	\item 第一级列表
	\item 第一级列表
	\item 第一级列表
\end{itemize}

\section{参考文献测试}
参考文献推荐大家统一使用文献管理工具Zotero或者JabRef,对文献进行统一管理,再也不用担心一条一条的交叉引用导致的错误啦,具体用法还是要靠大家自己去琢磨。D

测试一下引用\cite{selvaraju_grad-cam:_2016},引用以下gradcam这篇论文,连续两个脚注的使用{\footnote{awsl}}{\footnote{awms}},那么阿伟到底死没死。

\section{插图与表格测试}
\subsection{插图测试}
如\ref{fig:first_image_tset}是对此模版的第一张插图测试。

\begin{figure}[htbp] % h 当前位置,将图形放在正文文本中给出图形的位置,如果页面不够则不起作用;t 将图形放在页面顶部;b 将图形放在页面底部;p 浮动页
	% 在考虑这些参数的时候总以h-t-b-p来确定位置
	\centering % 格式居中
	\includegraphics[width = 0.5\linewidth]{./figures/intro/Chapter1.png}
	% 此处取消双标题的用法,在当前模板中只使用单标题 
	\caption{第一张插图测试\cite{selvaraju_grad-cam:_2016}} % 在图形的标题中中还可以进行引用
	\label{fig:first_image_tset}
\end{figure}

\subsection{表格测试}
在这里推荐制表采用功能强大的tabu宏包以取代其它制表宏包。具体tabu宏包的使用说明参见tabu宏包的说明文档。

以下节分别用来测试各种表格环境如,tabular,tabu,longtabu等,还有对caption格式的修改和测试。以下表格样式全部采用三线表。使用任意一种均可,这里推荐使用第一种和第二种,因为第三种我也没有整明白。

\subsubsection{array宏包tabular表格环境测试}
如\ref{tab:first_table_test}是对array宏包的tabular表格环境测试。
\begin{table}[htbp]
	\centering
	\caption{这是一个用tabular环境的测试用的表格}\label{tab:first_table_test}
	\begin{tabular}{lrr} 
		\toprule
		\textbf{行星}     & \textbf{赤道半径}km & \textbf{公转周期}d \\
		\midrule
		水星     & 2.439  & 87.9 \\
		金星     & 6.1    & 224.682 \\
		地球     & 6378.14 & 365.24 \\
		\bottomrule
	\end{tabular}%
\end{table}

\subsubsection{tabu宏包表格环境测试}
如\ref{tab:tabu_test_1}是对tabu宏包的tabu表格环境测试。在这里表格命令与\ref{tab:first_table_test}的命令相同,只是tabular环境改成了tabu环境。
\begin{table}[htbp]
	\centering
	\caption{这是一个用tabu环境的测试用的表格}\label{tab:tabu_test_1}
	\begin{tabu}{lrr}
		\toprule
		\textbf{行星}     & \textbf{赤道半径}km & \textbf{公转周期}d \\
		\midrule
		水星     & 2.439  & 87.9 \\
		金星     & 6.1    & 224.682 \\
		地球     & 6378.14 & 365.24 \\
		\bottomrule
	\end{tabu}%
\end{table}

\ref{tab:tabu_test_2}对tabu to表格的x列模式进行测试。在表格导言区中设置为X[1]X[2]X[2],表示这三列表格的列宽比值为1:2:2,总的表格宽度由tabu to环境设置,这里设置为0.6\textbackslash linewidth。相比于tabular环境,tabu环境的列宽设置方便许多。
\begin{table}[htbp]
	\centering
	\caption{tabu环境测试表格---X列模式}\label{tab:tabu_test_2}
	\begin{tabu} to 0.6\linewidth{X[1]X[2]X[2]}
		\toprule
		\textbf{行星}     & \textbf{赤道半径}km & \textbf{公转周期}d \\
		\midrule
		水星     & 2.439  & 87.9 \\
		金星     & 6.1    & 224.682 \\
		地球     & 6378.14 & 365.24 \\
		\bottomrule
	\end{tabu}%
\end{table}

如\ref{tab:tabu_test_3}是longtabu环境测试表格。longtabu环境不能用在table浮动体环境中。根据GB/T 7713.1-2006规定:如果某个表需要转页接排,在随后的各页上应重复表的编号。编号后跟标题(可省略)和“(续)”,置于表上方。续表应重复表头。

特别需要注意的是,longtabu是基于longtable宏包开发的,所以在nefuthesis.cls文件中已经插入了longtable宏包。longtable环境的所有功能都可以在longtabu中使用,如\textbackslash endhead,\textbackslash endfirsthead,\textbackslash endfoot,\textbackslash endlastfoot,和\textbackslash caption等。具体用法请参见longtable和tabu宏包的相应文档。
\begin{longtabu}{lccc}
	\caption{材料弹性模量及泊松比}\label{tab:tabu_test_3}\\
	\toprule
	名  称   & 弹性模量E/Gpa & 切变模量G/Gpa & 泊松比$\mu$ \\
	\midrule%
	\endfirsthead
	\caption{材料弹性模量及泊松比(续)}\\
	\toprule
	名  称   & 弹性模量E/Gpa & 切变模量G/Gpa & 泊松比$\mu$ \\
	\midrule%
	\endhead
	\bottomrule%
	\endfoot
	镍铬钢、合金钢 & 206    & 79.38  & 0.3 \\
	碳 钢    &  196~206 & 79     & 0.3 \\
	铸 钢    &  172~202 &        & 0.3 \\
	球墨铸铁   &  140~154 &  73~76 & 0.3 \\
	灰铸铁、白口铸铁 &  113~157 & 44     &  0.23~0.27 \\
	冷拔纯铜   & 127    & 48     &   \\
	轧制磷青铜  & 113    & 41     &  0.32~0.35 \\
	轧制纯铜   & 108    & 39     &  0.31~0.34 \\
	轧制锰青铜  & 108    & 39     & 0.35 \\
	铸铝青铜   & 103    & 41     & 0.3 \\
	冷拔黄铜   &  89~97 &  34~36 &  0.32~0.42 \\
	轧制锌    & 82     & 31     & 0.27 \\
	硬铝合金   & 70     & 26     & 0.3 \\
	轧制铝    & 68     &  25~26 &  0.32~0.36 \\
	铅      & 17     & 7      & 0.42 \\
	玻璃     & 55     & 22     & 0.25 \\
	混凝土    &  14~39 &  439~15.7 &  0.1~0.18 \\
	纵纹木材   &  9.8~12 & 0.5    &   \\
	横纹木材   &  0.5~0.98 &  0.44~0.64 &   \\
	橡胶     & 0.00784 &        & 0.47 \\
	电木     &  1.96~2.94 &  0.69~2.06 &  0.35~0.38 \\
	赛璐珞    &  1.71~1.89 &  0.69~0.98 & 0.4 \\
	可锻铸铁   & 152    &        &  \\
	拔制铝线   & 69     &        &  \\
	大理石    & 55     &        &  \\
	花岗石    & 48     &        &  \\
	石灰石    & 41     &        &  \\
	尼龙1010 & 1.07   &        &  \\
	夹布酚醛塑料 &  4~8.8 &        &  \\
	石棉酚醛塑料 & 1.3    &        &  \\
	高压聚乙烯  &  0.15~0.25 &        &  \\
	低压聚乙烯  &  0.49~0.78 &        &  \\
	聚丙烯    &  1.32~1.42 &        &  \\
	硬聚氯乙烯  &  3.14~3.92 &        &  \\
	聚四氟乙烯  &  1.14~1.42 &        &  \\
\end{longtabu}%

\subsection{子图}
这里子图的排版推荐使用subcaption宏包,不再推荐使用subfig宏包,更不推荐使用subfigure宏包。值得注意的是,在nefuthesis.cls文件中已经写入了subcaption宏包,而且subcaption宏包与subfigure和subfig宏包是相互冲突的。因此,如果你还想使用subfig宏包而不想使用subcaption宏包,请自己到nefuthesis.cls文件的相关位置更改,具体的使用及修改方法参见相应的宏包说明文档。不过在这里还是不推荐直接去更改nefuthesis.cls文档,除非你对\LaTeX 的相关命令很清楚,知道自己在改什么,并且不会对其他格式产生影响。

具体的subcaption宏包使用方法我这里不仔细介绍,以下只是对subcaption进行一些简单的测试,主要是格式调整和交叉引用。

如\ref{fig:subfig_test1}是有两张子图的模式,对子图进行交叉引用,如\ref{subfig:1a}和\ref{subfig:1b}。
\begin{figure}[htbp]
	\centering
	\begin{subfigure}[b]{.4\textwidth}
		\centering
		\includegraphics[width = \textwidth]{./figures/intro/Chapter2.png}
		\subcaption{书籍排版与普通排版}\label{subfig:1a}
	\end{subfigure}
	\quad
	\begin{subfigure}[b]{.4\textwidth}
		\centering
		\includegraphics[width = \textwidth]{./figures/intro/Chapter3.png}
		\caption{\TeX 的控制系列}\label{subfig:1b}
	\end{subfigure}
	\caption{子图模式测试1:2张图}\label{fig:subfig_test1}
\end{figure}

如\ref{fig:subfig_test2}是有四张子图的模式,对子图进行交叉引用,如\ref{subfig:2a}、\ref{subfig:2b}、\ref{subfig:2c}和\ref{subfig:2d}。

\begin{figure}[htbp]
	\centering
	\begin{subfigure}[b]{.4\textwidth}
		\centering
		\includegraphics[width = \textwidth]{./figures/intro/Chapter4.png}
		\caption{字体}\label{subfig:2a}
	\end{subfigure}
	\begin{subfigure}[b]{.4\textwidth}
		\centering
		\includegraphics[width = \textwidth]{./figures/intro/Chapter5.png}
		\caption{编组}\label{subfig:2b}
	\end{subfigure}
	\begin{subfigure}[b]{.4\textwidth}
		\centering
		\includegraphics[width = \textwidth]{./figures/intro/Chapter6.png}
		\caption{运行\TeX}\label{subfig:2c}
	\end{subfigure}
	\begin{subfigure}[b]{.4\textwidth}
		\centering
		\includegraphics[width = \textwidth]{./figures/intro/Chapter7.png}
		\caption{\TeX 工作原理}\label{subfig:2d}
	\end{subfigure}
	\caption{子图模式测试2:4张图}\label{fig:subfig_test2}
\end{figure}

\subsection{数学模式测试}
数学模式测试,主要测试数学字体,编号和交叉引用。这里首先推荐使用\texttt{align}和\texttt{align*}数学模式环境,大多数行间数学模式只需要用这个环境就可以了。

交叉引用测试,如交引用命令{\ttfamily \textbackslash eqref}和\texttt{\textbackslash ref}命令的区别。如公式\eqref{eq:test1},公式\ref{eq:test1}显示,\texttt{\textbackslash eqref}命令比\texttt{\textbackslash ref}命令的应用结果多了个括号。

如公式\eqref{eq:test3}是单行公式环境,查看公式\eqref{eq:test3}和\eqref{eq:test1}之间的区别,好像在单行公式中没什么区别。
\begin{align}\label{eq:test3}
f(x) = 2(x + 1)^{2} - 1
\end{align}

\texttt{align}公式环境,用在单行中。
\begin{align}\label{eq:test1}
f(x) = 2(x + 1)^{2} - 1
\end{align}

在这里,中间插入一些文字以形成段落,查看行间公式与上下文之间的间隙。
\begin{align*}
f(x) = 2(x + 1)^{2} - 1
\end{align*}
在这里,中间插入一些文字以形成段落,查看行间公式与上下文之间的间隙。下一个公式\eqref{eq:test2}是一个公式组,它在“=”位置对齐。
\begin{align}\label{eq:test2}
f(x) & = 2(x + 1)^{2} - 1\\
& = 2(x^{2} + 2x +1)-1\\
& = 2x^{2} + 4x + 1
\end{align}

\section{关于引用}
图表的引用通过{\ttfamily \textbackslash autoref} 命令即可,使用ST LaTeXTools 插件还能自动补全。如果要修改前缀,那么就用{\ttfamily \textbackslash recnewcommand \textbackslash figureautorefname\{好图\}}即可,详见hyperref宏包说明。

\section{出现的问题}
\subsection{\textbackslash texttt}
在这里发现一个问题,在下面的例子中可以发现,在中文中使用\textbackslash texttt\{\}命令时,前面的汉字与接下来的英文单词的空隙明显比接下来单词跟汉字的间隙要大,但是其它命令没有什么问题。

\begin{center}
	\noindent 问题\texttt{问题}问题,问题\textbackslash\texttt{问题}问题。\\
	问题\texttt{ref} 问题,问题\texttt{\textbackslash ref} 问题。\\
	问题\textbf{ref}问题,问题\textbf{\textbackslash ref}问题。\\
	问题\textsf{ref}问题,问题\textsf{\textbackslash ref}问题。\\
	problem \texttt{ref} problem,problem \texttt{\textbackslash ref} problem.\\
	problem \textbf{ref} problem,problem \textbf{\textbackslash ref} problem.\\
	problem \textsf{ref} problem,problem \textsf{\textbackslash ref} problem.
\end{center}

原来的编译环境为texlive 2014,编译环境改为texlive 2015后,问题解决。

\backmatter
% 结论


\begin{conclusion}
	我是结论......
\end{conclusion}
% 参考文献 这里的文件为bibtex格式 使用文中提到的 zotero jabref文献管理工具可以轻松将bibtex文件导出
\bibliography{references}
% 攻读学位期间发表的论文

\chapter{攻读学位期间发表的学术论文}
\begin{thebibliography}{00} % 这里的00不能去掉,目前还没搞清楚是因为什么, 这里的使用的文献格式也可以轻松从文献管理工具导出->粘贴至复制板而不是bibtex
	\bibitem{Song:smart_cities}
	Houbing Song, Ravi Srinivasan, Tamim Sookoor, Sabina Jeschke, Smart Cities: Foundations, Principles and Applications. ISBN: 978-1-119-22639-0, Hoboken, NJ: Wiley, 2017, pp.1-906.
\end{thebibliography}
% 致谢
% !TEX root = ../thesis.tex
\chapter{致\texorpdfstring{\NEFUspace}{}谢}
我是致谢......

\vspace{2cm}
\hfill
% 接下来这个东西可以考虑根据实际需求注释掉
\begin{minipage}{14em}
\begin{center}
于XX地\quad 225年1月1日\\
我的名
\end{center}
\end{minipage}

\end{document}